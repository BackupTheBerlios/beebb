%\documentclass[11pt,oneside]{article}
\documentclass[12pt,a4paper,twoside]{article}

\RequirePackage[T1]{fontenc}
\RequirePackage{times}
%\usepackage[T1]{fontenc}
\RequirePackage[latin2]{inputenc}
\RequirePackage[polish]{babel}
%\usepackage[polish]{babel}

\RequirePackage{comment}

\RequirePackage{a4wide}
\RequirePackage{longtable}
\RequirePackage{multicol}
\RequirePackage{url}
\usepackage[dvips]{color,graphicx}
%\usepackage[latin2]{inputenc}

\author{Marcin Pruszczy"nski}
\title{Struktura repozytorium CVS}
\date{\today}

\begin{document}

\tableofcontents

\section{CVS}
Projekt Bee posiada swoje repozytorium na serwerze CVS. Adres serwera to {\it wilk.waw.pl}. Autoryzacja opiera si� o {\it pserver}, kt�ry dost�pny jest na standardowym numerze portu 2401. �cie�ka do repozytorium to: {\it /home/cvs/projekty}.
Struktura katalog�w w repozytorium:
\begin{itemize}
\item {\it baza\_danych} -- katalog przeznaczony na skrypty tworz�ce bazy danych
\item {\it dokumentacja} -- katalog przeznaczony dla plik�w sk�adaj�cych si� na dokumentacj� projektu
\item {\it layout} -- katalog zawieraj�cy szkielet wygl�du projektu. Katalog i zawarto�� jest nieaktualna.
\item {\it program} -- katalog zawieraj�cy projekt Bee b�d�cy projektem �rodowiska NetBeans w wersji 4.1
\item {\it teksty} -- katalog przeznaczony na pliki zawieraj�ce opisy zagadnie�, kt�re mog� by� przydatne podczas implementacji projektu
\item {\it testy} -- katalog przeznaczony na pliki
\end{itemize}

\end{document}
