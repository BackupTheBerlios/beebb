\documentclass[12pt,a4paper,twoside]{article}
\usepackage[T1]{fontenc}
\RequirePackage{times}
\usepackage[latin2]{inputenc}
\usepackage[polish]{babel}
\RequirePackage{comment}

\RequirePackage{a4wide}
\RequirePackage{longtable}
\RequirePackage{multicol}
\RequirePackage{url}
%\usepackage[dvips]{color,graphicx}
%\usepackage{graphicx}


\begin{document}
%\author{Pawe"l Boguszewski \and Marcin Pruszczy"nski \and Piotr Sobczak}
%\title{Przegl"ad program"ow -- fora dyskusyjne -- pod wzgl"edem ich funkcjonalno"sci.}
%\subtitle{Programy konkurencyjne dla projektu programistycznego forum dyskusyjne.}
%\date{\today}

\begin{center}
\Huge{Przegl"ad program"ow -- fora dyskusyjne -- pod wzgl"edem ich funkcjonalno"sci.}
\newline
\newline
\newline
\newline
\end{center}

\begin{center}
\Large{Programy konkurencyjne dla projektu programistycznego forum dyskusyjne.}
\newline
\newline
\newline
\newline
\newline
\newline
\end{center}


\begin{center}
Pawe"l Boguszewski
\end{center}
\begin{center}
Marcin Pruszczy"nski
\end{center}
\begin{center}
Piotr Sobczak
\end{center}
\begin{center}
~
\newline
\newline
\newline
\newline
\newline
\newline
\end{center}


\begin{center}
Luty 2005
%\maketitle
\end{center}

\newpage

\section{PhpBB}
Jedno z najpopularniejszych for dost"epnych w internecie oparte na j"ezyku skryptowym PHP. G"l"owny nacisk po"lo\.zono na modularyzacj"e, bezpiecze"nstwo, wieloj"ezyczno"s"c oraz obs"lug"e wielu baz danych. PhpBB mo\.zna dostosowa"c dowolnie do wygl"adu strony poprzez wsparcie dla motyw"ow, przez co forum mo\.ze wygl"ada"c bardzo profesjonalnie . tak jakby by"lo docelowo napisane dla danego serwisu. Ka\.zdy zarejestrowany u\.zytkownik posiada sw"oj profil, dzi"eki kt"oremu ma mo\.zliwo"s"c personalizacji wygl"adu. U\.zytkownicy mog"a porozumiewa"c si"e ze sob"a poza forum publicznym za pomoc"a prywatnych wiadomo"sci. Administrator dostaje do dyspozycji szczeg"o"lowy panel sterowania -- dzi"eki kt"oremu mo\.ze tworzy"c nowe fora, zarz"adza"c u\.zytkownikami a tak\.ze tworzy"c kopie zapasow"a oraz ustawia"c wygl"ad forum.\\
Jako jeden z celi jest kompatybilno"s"c wstecz -- mo\.zna by"c pewnym \.ze aktualn"a wersj"e phpBB uda si"e bezproblemowo uaktualni"c w przysz"lo"sci.  
	
Charakterystyka:
\begin{itemize}
\item	Szybki i prosty proces instalacji, pozwalaj"acy uzyska"c dzia"laj"ace forum w kilka minut.
\item	Wsparcie automatycznej aktualizacji z wcze"sniejszej wersji.
\item	Wsp"o"lpraca z wieloma bazami danych:
\begin{itemize}
    \item	MySQL
    \item	PostgreSQL
    \item	Microsoft SQL Server
    \item	Microsoft Access 
\end{itemize}
\item	Bezpiecze"nstwo jako priorytet.
\item	Has"la przechowywane w postaci zakodowanej w bazie danych.
\item	Wsparcie dla ciasteczek oraz sesji opartych na URI.
\item	Mo\.zliwo"s"c tworzenia prywatnych for dla wybranej grupy u\.zytkownik"ow.
\item	Obs"luga zaawansowanego wyszukiwania.
\item	Rozbudowane opcje przy komponowaniu nowych wiadomo"sci -- mo\.zliwo"s"c cytowania, wy"swietlania kodu \'zr"od"lowego, wklejania obrazk"ow, dodawania automatycznych link"ow, zmiany czcionki.
\item	Wsparcie dla znacznik"ow HTML oraz emotikonek przy tworzeniu nowych wiadomo"sci.
\item	Powiadamianie o zmianach w wybranych w"atkach.
\item	Mo\.zliwo"s"c grupowania u\.zytkownik"ow
\item	Nadawanie praw odczytu/zapisu/moderacji dla grup i/lub u\.zytkownik"ow
\item	Mo\.zliwo"s"c wprowadzenia rang w zale\.zno"sci od aktywno"sci na forum
\item	Ka\.zdy zarejestrowany u\.zytkownik ma sw"oj profil, gdzie ustawia swoje dane oraz preferencje co do wygl"adu forum.
\item	Wbudowany system komunikacji pomi"edzy u\.zytkownikami.
\item	Mo\.zliwo"s"c moderacji wypowiedzi.
\item	Rozbudowana mo\.zliwo"s"c administracji forum w postaci blokowania u\.zytkownik"ow (po nazwie lub adresie ip), cenzorowania wypowiedzi, tworzenia kopii zapasowych.
\item	R"o\.znorodno"s"c motyw"ow z mo\.zliwo"sci"a tworzenia w"lasnych.
\item	Wygl"ad oparty na css.
\item	Forum zgodne z XHTML 1.0 i HTML 4.01
\item	Bezp"latne.
\end{itemize}

\newpage	    
\section{Invision Power Board}
IPB jest komercyjnym produktem skierowanym do wymagaj"acych u\.zytkownik"ow. Posiada zintegrowany instalator oraz mechanizmy pozwalaj"ace skonwertowa"c konkurencyjne forum do IPB. Forum poprzez obs"lug"e motyw"ow mo\.zna "latwo dostosowa"c do wygl"adu serwisu.
	    
Charakterystyka:
\begin{itemize}	    
\item	    Wbudowany instalator oraz wsparcie dla RDBMS
\item	    Mo\.zliwo"s"c ignorowania lub blokowania u\.zytkownik"ow.
\item	    "Latwa integracja ze stron"a WWW.
\item	    Wsparcie dla wieloj"ezyczno"s"c z mo\.zliwo"sci"a rozszerzenia o dowolny j"ezyk.
\item	    Wbudowany edytor CSS.
\item	    Rozbudowany mechanizm wyszukiwania.
\item	    Edytor wiadomo"sci pozwalaj"acy na u\.zywanie emotikonek, znacznik"ow HTML, a tak\.ze wielopoziomowego cytowania.
\item	    Mo\.zliwo"s"c cenzorowania wypowiedzi poprzez list"e zakazanych s"l"ow.
\item	    Wsparcie dla profili u\.zytkownika.
\item	    Grupowanie u\.zytkownik"ow.
\item	    Mo\.zliwo"s"c ustawienia zezwole"n dla u\.zytkownik"ow.
\item	    Blokowanie forum has"lem.
\item	    Moderacja wypowiedzi przez wybranych u\.zytkownik"ow.
\item	    Mo\.zliwo"s"c generacji statystyk.
\item	    Mo\.zliwo"s"c eksportu tematu do pliku Word/HTML.
\item	    Archiwum w pliku XML.
\item	    Mo\.zliwo"s"c ukazania w"atku z minimaln"a dekoracj"a w celu wydrukowania.
\end{itemize}	    

\newpage
\section{YaBB}
YaBB jest forum rozwijanym przez spo"leczno"s"c open-source. Posiada zalety for"ow komercyjnych. Zintegrowany instalator pozwala na szybk"a i bezproblemow"a instalacj"e produktu. YaBB wspiera wiele platform -- od Linuksa po Macosx'a i Windowsa, bardzo rozbudowane "srodowisko u\.zytkownik"ow oferuje swoj"a pomoc przy instalacji forum na wymienionych systemach.\\
Ka\.zdy u\.zytkownik posiada sw"oj profil -- dzi"eki kt"oremu mo\.ze zpersonalizowa"c wygl"ad forum, a tak\.ze umie"sci"c swoje dane. 
		
Charakterystyka:
\begin{itemize}
\item		Powiadamianie w postaci szybkich wiadomo"sci.
\item		Menu do nawigacji.
\item		Indeks forum.
\item		Statystyki zawieraj"ace: nowych u\.zytkownik"ow, og"oln"a liczb"e wiadomo"sci oraz najnowsze wiadomo"sci.
\item		Graficzne powiadomienia o nowych wiadomo"sciach w w"atku
\item		Narz"edzia wspomagaj"ace wpisywanie wiadomo"sci w postaci sprawdzania pisowni oraz mo\.zliwo"s"c stosowania znacznik"ow HTML
\item		Mo\.zliwo"s"c ustawiania powiadamiania o zmianach w wybranych w"atkach.
\item		Wbudowany system komunikacji pomi"edzy u\.zytkownikami
\item		Panel administracyjny daj"acy mo\.zliwo"s"c pe"lnej kontroli i moderacji z poziomu forum.
\item		Mo\.zliwo"s"c tworzenia prywatnych for dla wybranej grupy u\.zytkownik"ow.
\item		Ka\.zdy zarejestrowany u\.zytkownik ma sw"oj profil -- pozwalaj"acy na personalizacj"e wygl"adu forum.
\item		emotikonki z mo\.zliwo"sci"a ich wy"l"aczenia w zale\.zno"sci od ustawie"n w profilu u\.zytkownika.
\item		Szablony wygl"adu oparte na CSS z mo\.zliwo"sci"a modyfikacji istniej"acych oraz dodawania nowych.
\item		Pomoc on-line dla u\.zytkownik"ow i administrator"ow.
\item		Szybka i prosta rejestracja u\.zytkownik"ow z generatorem losowych hase"l.
\item		Mo\.zliwo"s"c przenoszenia w"atk"ow przez moderator"ow lub administrator"ow.
\end{itemize}		

\newpage		
\section{Podsumowanie}
Powy\.zsze przyk"lady prezentuj"a jak bardzo podobn"a funkcjonalno"sci"a charakteryzuj"a si"e dost"epne na rynku fora. Wsp"olnie opracowany standard obowi"azuje obecnie prawie we wszystkich produktach. Wydaje si"e, \.ze \.zaden z program"ow  na rynku nie odznacza si"e prze"lomow"a funkcjonalno"sci"a. Na uwag"e zas"luguje polskie forum txtBB, kt"ore zamiast bazy danych u\.zywa plik"ow tekstowych do zapisywania archiwum wiadomo"sci. Czytaj"ac opinie u\.zytkownik"ow takie rozwi"azanie wcale nie jest gorsze pod wzgl"edem wydajno"sci, a pozwala zainstalowa"c forum w bardzo skromnych warunkach "srodowiskowych -- bez potrzeby u\.zycia bazy danych.

\end{document}
		    
		    
		    